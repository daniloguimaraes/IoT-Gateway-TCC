% abtex2-modelo-artigo.tex, v-1.9.2 laurocesar
% Copyright 2012-2014 by abnTeX2 group at http://abntex2.googlecode.com/ 
%

% ------------------------------------------------------------------------
% ------------------------------------------------------------------------
% abnTeX2: Modelo de Artigo Acadêmico em conformidade com
% ABNT NBR 6022:2003: Informação e documentação - Artigo em publicação 
% periódica científica impressa - Apresentação
% ------------------------------------------------------------------------
% ------------------------------------------------------------------------

\documentclass[
    % -- opções da classe memoir --
    article,            % indica que é um artigo acadêmico
    11pt,               % tamanho da fonte
    oneside,            % para impressão apenas no verso. Oposto a twoside
    a4paper,            % tamanho do papel. 
    % -- opções da classe abntex2 --
    %chapter=TITLE,     % títulos de capítulos convertidos em letras maiúsculas
    %section=TITLE,     % títulos de seções convertidos em letras maiúsculas
    %subsection=TITLE,  % títulos de subseções convertidos em letras maiúsculas
    %subsubsection=TITLE % títulos de subsubseções convertidos em letras maiúsculas
    % -- opções do pacote babel --
    english,            % idioma adicional para hifenização
    brazil,             % o último idioma é o principal do documento
    sumario=tradicional
    ]{abntex2}


% ---
% PACOTES
% ---

% ---
% Pacotes fundamentais 
% ---
\usepackage{lmodern}            % Usa a fonte Latin Modern
\usepackage[T1]{fontenc}        % Selecao de codigos de fonte.
\usepackage[utf8]{inputenc}     % Codificacao do documento (conversão automática dos acentos)
\usepackage{indentfirst}        % Indenta o primeiro parágrafo de cada seção.
\usepackage{nomencl}            % Lista de simbolos
\usepackage{color}              % Controle das cores
\usepackage{graphicx}           % Inclusão de gráficos
\usepackage{microtype}          % para melhorias de justificação
\usepackage{listings}           % formatar codigo fonte http://linorg.usp.br/CTAN/macros/latex/contrib/listings/listings.pdf
% ---
        
% ---
% Pacotes adicionais, usados apenas no âmbito do Modelo Canônico do abnteX2
% ---
\usepackage{lipsum}             % para geração de dummy text
% ---
        
% ---
% Pacotes de citações
% ---
\usepackage[brazilian,hyperpageref]{backref}     % Paginas com as citações na bibl
\usepackage[alf]{abntex2cite}   % Citações padrão ABNT
% ---

% ---
% Configurações do pacote backref
% Usado sem a opção hyperpageref de backref
\renewcommand{\backrefpagesname}{Citado na(s) página(s):~}
% Texto padrão antes do número das páginas
\renewcommand{\backref}{}
% Define os textos da citação
\renewcommand*{\backrefalt}[4]{
    \ifcase #1 %
        Nenhuma citação no texto.%
    \or
        Citado na página #2.%
    \else
        Citado #1 vezes nas páginas #2.%
    \fi}%
% ---

% ---
% Informações de dados para CAPA e FOLHA DE ROSTO
% ---
\titulo{Construindo um gateway para dispositivos IoT com MEAN stack e MQTT}
\autor{Danilo Guimarães \thanks{guimaraesdjl@gmail.com} \and Ricardo Rodrigues \thanks{ricardo.faria@outlook.com.br} \and Uillan Araújo \thanks{uillan@outlook.com}}
\local{Brasil}
\instituicao{Centro Universitário Alves Faria}
\data{2017, v-1.9.2}


% ---

% ---
% Configurações de aparência do PDF final

% alterando o aspecto da cor azul
\definecolor{blue}{RGB}{41,5,195}

% informações do PDF
\makeatletter
\hypersetup{
        %pagebackref=true,
        pdftitle={\@title}, 
        pdfauthor={\@author},
        pdfsubject={\@title},
        pdfcreator={LaTeX with abnTeX2},
        pdfkeywords={abnt}{latex}{abntex}{abntex2}{atigo científico}, 
        colorlinks=true,            % false: boxed links; true: colored links
        linkcolor=blue,             % color of internal links
        citecolor=blue,             % color of links to bibliography
        filecolor=magenta,              % color of file links
        urlcolor=blue,
        bookmarksdepth=4
}
\makeatother
% --- 

% ---
% compila o indice
% ---
\makeindex
% ---

% ---
% Altera as margens padrões
% ---
\setlrmarginsandblock{3cm}{3cm}{*}
\setulmarginsandblock{3cm}{3cm}{*}
\checkandfixthelayout
% ---

% --- 
% Espaçamentos entre linhas e parágrafos 
% --- 

% O tamanho do parágrafo é dado por:
\setlength{\parindent}{1.3cm}

% Controle do espaçamento entre um parágrafo e outro:
\setlength{\parskip}{0.2cm}  % tente também \onelineskip

% Espaçamento simples
\SingleSpacing

% ----
% Início do documento
% ----
\begin{document}

% Retira espaço extra obsoleto entre as frases.
\frenchspacing 

% ----------------------------------------------------------
% ELEMENTOS PRÉ-TEXTUAIS
% ----------------------------------------------------------

%---
%
% Se desejar escrever o artigo em duas colunas, descomente a linha abaixo
% e a linha com o texto ``FIM DE ARTIGO EM DUAS COLUNAS''.
% \twocolumn[           % INICIO DE ARTIGO EM DUAS COLUNAS
%
%---
% página de titulo
\maketitle

% resumo em português
\begin{resumoumacoluna}
 Conforme a ABNT NBR 6022:2003, o resumo é elemento obrigatório, constituído de
 uma sequência de frases concisas e objetivas e não de uma simples enumeração
 de tópicos, não ultrapassando 250 palavras, seguido, logo abaixo, das palavras
 representativas do conteúdo do trabalho, isto é, palavras-chave e/ou
 descritores, conforme a NBR 6028. (\ldots) As palavras-chave devem figurar logo
 abaixo do resumo, antecedidas da expressão Palavras-chave:, separadas entre si por
 ponto e finalizadas também por ponto.
 
 \vspace{\onelineskip}
 
 \noindent
 \textbf{Palavras-chaves}: Internet of Things. IoT Gateway. Arquitetura de Software.
\end{resumoumacoluna}

\renewcommand{\resumoname}{Abstract}
\begin{resumoumacoluna}
	\begin{otherlanguage*}{english}
		According to ABNT NBR 6022:2003, an abstract in foreign language is a back
		matter mandatory element.
		
		\vspace{\onelineskip}
		
		\noindent
		\textbf{Key-words}: Internet of Things. IoT Gateway. Software Architecture.
	\end{otherlanguage*}  
\end{resumoumacoluna}

% ]                 % FIM DE ARTIGO EM DUAS COLUNAS
% ---

% ----------------------------------------------------------
% ELEMENTOS TEXTUAIS
% ----------------------------------------------------------
\textual

% ----------------------------------------------------------
%% CAPITULOS
% ----------------------------------------------------------

\section*{Introdução}
\addcontentsline{toc}{section}{Introdução}
\label{sec:intro}

Com os recentes avanços das tecnologias, especificamente nas últimas décadas e devido a democratização da Internet, nossa sociedade tem caminhado para um cenário cada vez mais conectado. Se antes apenas super-computadores e máquinas robustas eram conectadas à rede, a tendência nos próximos anos é que dispositivos cada vez menores também tenham seu espaço na Internet.
A essa tendência chamamos \textit{Internet of Things}, ou simplesmente IoT. É uma nova visão que descreve objetos fazendo parte da rede, onde cada um deles é unicamente identificado, acessível através da rede, com posição e estado conhecido, captando informações sensoriais ou agindo sobre o ambiente. Serviços são construídos com base nesses objetos. Estima-se que até 2020, sejam investidos cerca de US\$ 267 bi na indústria e serviços voltados para IoT \cite{BCGPerspectives,Forbes}.

A principal motivação da realização desse trabalho foi adentrar no assunto de IoT para entender melhor como as tecnologias envolvidas funcionam. Portanto, o objetivo deste trabalho é construir um Smart IoT Gateway open-source, funcional e utilizável em projetos de pequeno e médio porte. Ser um smart gateway de uso simples, onde usuários cadastrem ações baseados nos dados enviados por um sensor cadastrado. Neste trabalho não temos como objetivo construir um projeto de hardware para um Gateway IoT e nem competir em quaisquer aspecto com soluções existentes no mercado.

\section{Internet of Things}
\label{sec:iot}

-- Escrever direto sobre IOT
\section{IoT Gateway}
\label{sec:iotGateway}

-- Escrever direto sobre Gateway IOT
\section{Solução desenvolvida}
\label{sec:iotGateway}

Esta seção detalha os objetivos e decisões arquiteturais. Foi desenvolvida uma solução de software de \textit{Smart Gateway IoT} que está disponibilizada no Github, capaz de receber dados através de uma rede utilizando o protocolo MQTT	(\ref{mqtt}) de um dispositivo previamente cadastro e armazenar essas informações. Os dados recebidos são analisados para a estrutura que irá definir a execução ou não de um fluxo de notificação através de SMS (\ref{sms}) para um número definido.

\subsection{Tecnologias Utilizadas} 
\begin{itemize}
	\item Node.js \footnote{NodeJS - \url{https://nodejs.org/en/}} v6.x;
	\item TypeScript \footnote{Typescript - \url{http://www.typescriptlang.org/}} v2.3 com transpile para ES6;
	\item TSLint \footnote{TSLint - \url{https://palantir.github.io/tslint/}} v4.x com recomendações gerais padrão;
	\item Jest \footnote{Jest - \url{https://facebook.github.io/jest/}}para teste unitário e cobertura;
	\item AngularJS \footnote{AngularJS - \url{https://angularjs.org/}} v1.6 para o front end da aplicação ;
	\item MongoDB \footnote{MongoDB - \url{https://www.mongodb.com/}} para persistência;
	\item MQTT (\ref{mqtt}) como protocolo de comunicação entre os sensores e o Gateway;
	\item SMS (\ref{sms}) como serviço de mensageria;
	\item ExpressJS \footnote{ExpressJS - \url{http://expressjs.com/}} como framework web Node.js 
\end{itemize}

O Node.js foi escolhido por conta de seu baixo consumo de memória e processamento, além da sua característica de Non-Blocking IO \cite{NodeJSNonBlockingIO}, garantindo que possamos servir mais clientes com menos recursos, objetivo essencial para aplicações que podem ser executadas em um Raspberry Pi por exemplo.

A escolha pelo TypeScript \footnote{Typescript - \url{http://www.typescriptlang.org/}}, linguagem que é um superset do Javascript padrão foi motivada por garantir uma estrutura tipada, de forma que a manutenção do código fosse facilitada e as regras de negócio pudessem estar ligadas a um contrato de objeto.

Já o AngularJS \footnote{AngularJS - \url{https://angularjs.org/}} foi escolhido, por ser uma tecnologia que funciona com javascript nativo, sem necessidade de nenhum pós-processador para servir a aplicação aos clientes, permitindo o seu uso diretamente entre os arquivos estáticos do mesmo servidor Node.js que expõe a aplicação. Além disso, contou como um ponto para a escolha, a experiência prévia da equipe no desenvolvimento com esta tecnologia.

\subsubsection{Protocolos de comunicação}
\label{protocolos}

\phantomsection{SMS}\label{sms}, abreviação de \textit{Short Message Service}, é um serviço de troca de mensagens curtas de textos que permite o envio de mensagens para aparelhos celulares, conforme os padrões definidos no GSM (\textit{Global System for Mobile Communications}). Seu uso é bastante popular (cerca de 3,6 bilhões de usuários) \cite{SMS} e ubíquo (presente em praticamente qualquer país) \cite{SMSItu}. Além do que seu uso é fácil e barato, uma vez que o usuário receptor não requer conexão com a Internet para receber a mensagem, bastando possuir sinal com a rede de telefonia.

O \phantomsection{MQTT}\label{mqtt}, abreviação de \textit{Message Queue Telemetry Transport}, é um protocolo de comunicação altamente voltado para IoT. Ele foi arquitetado para ser um sistema de mensageria leve do tipo publisher/subscriber, para rodar em dispositivos limitados, tanto do ponto de vista da quantidade de memória para execução do programa, quanto do ponto de vista da conectividade. Redes lentas ou com alta latência não são problemas para esse protocolo. Geralmente, sensores IoT podem residir em locais extremamente hostis do ponto de vista de infra-estrutura de conexão de dados.

\subsection{Modelo de Dados}
O modelo de dados foi concebido com a intenção de tornar as etapas do processo altamente plugáveis e customizáveis no curto e longo prazo, garantindo as funcionalidades básicas do MVP \footnote{\textit{Minimum Viable Product}, ou Produto Mínimo Viável, é um termo utilizado para caracterizar um produto com as features necessárias para os primeiros clientes \cite{MVP}} executado e a possibilidade de extensibilidade no futuro com retrocompatibilidade. 

Nesta modelagem, a entidade \verb|Dispositivo| identifica um aparelho qualquer que se conecte ao gateway para transmitir informações, o aparelho deve definir um ID de conexão que é cadastrado nesta entidade. Com relação direta a entidade \verb|Dispositivo|, e a entidade \verb|Trigger| modela um gatilho clássico, composto de condição e ação. Nesta perspectiva, a condição é a entidade \verb|Operação| que permite avaliar os valores recebidos e identificar se eles são iguais ou estão no intervalo de valor definidos por esta entidade. A entidade \verb|Evento| por sua vez é a ação, onde é definido o que acontecerá caso a condição seja atendida, por exemplo, o envio de SMS para um destinatário com o determinado texto, ver Figura~\ref{fig:modeloDeDados}.

Além das entidades que servem diretamente a execução do fluxo principal da aplicação, temos 3 entidades de funções secundárias. \verb|Configuracao| é responsável por armazenar os dados utilizados para configurar a aplicação, como a configuração das credenciais para uso do serviço de SMS. As entidades \verb|HistoricoEvento| e \verb|HistoricoMensagem| são responsáveis por armazenar os dados recebidos e os eventos executados.

\begin{figure}[h!]
	\begin{center}
		\includegraphics[width=1.085\textwidth]{./img/modelo-de-dados}
		\caption{Representação esquemática da modelagem de dados.}
		\label{fig:modeloDeDados}
	\end{center}
\end{figure}

\begin{figure}[h!]
		\begin{center}
		\includegraphics[width=0.9\textwidth]{./img/fluxograma}
		\caption{Representação esquemática do fluxo Gateway IoT.}
		\label{fig:fluxograma}
	\end{center}
\end{figure}

A modelagem desenvolve uma estrutura sequencial que parte da identificação do dispositivo, análise dos gatilhos ligados a este dispositivo, avaliação da operação lógica e liberação para execução do evento, ver Figura~\ref{fig:fluxograma}.  

\subsection{Requisitos funcionais}
\label{reqFuncionais}

Os requisitos funcionais levados em consideração nesse trabalho foram:

\begin{itemize}
	\item Requisito 1: Envio de mensagens via SMS \cite{SMS}
	
	\item Requisito 2: Cadastro de número de celular.
\end{itemize}

\subsection{Requisitos não funcionais}
\label{reqNaoFuncionais}

Requisitos não funcionais, como os de extensibilidade e de retrocompatibilidade, agregam muito esforço no desenho de um MVP, mas são aspectos que em hipótese alguma podem ser desconsiderados.


% ---
% Finaliza a parte no bookmark do PDF, para que se inicie o bookmark na raiz
% ---
\bookmarksetup{startatroot}% 
% ---

% ---
% Conclusão
% ---
\section*{Considerações finais}
\addcontentsline{toc}{section}{Considerações finais}

\lipsum[1]

\begin{citacao}
\lipsum[2]
\end{citacao}

\lipsum[3]


% ----------------------------------------------------------
% ELEMENTOS PÓS-TEXTUAIS
% ----------------------------------------------------------
\postextual

% ----------------------------------------------------------
% Referências bibliográficas
% ----------------------------------------------------------
\clearpage
\bibliography{./bib/artigo}

% ----------------------------------------------------------
% Glossário
% ----------------------------------------------------------
%
% Há diversas soluções prontas para glossário em LaTeX. 
% Consulte o manual do abnTeX2 para obter sugestões.
%
%\glossary

% ----------------------------------------------------------
% Apêndices
% ----------------------------------------------------------

% ---
% Inicia os apêndices
% ---
\clearpage
\input{./pos/apendice}



% ----------------------------------------------------------
% Anexos
% ----------------------------------------------------------
\cftinserthook{toc}{AAA}
% ---
% Inicia os anexos
% ---
%\anexos
%\input{./anexos/anexos}

\end{document}