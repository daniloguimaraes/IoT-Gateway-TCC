% abtex2-modelo-artigo.tex, v-1.9.2 laurocesar
% Copyright 2012-2014 by abnTeX2 group at http://abntex2.googlecode.com/ 
%

% ------------------------------------------------------------------------
% ------------------------------------------------------------------------
% abnTeX2: Modelo de Artigo Acadêmico em conformidade com
% ABNT NBR 6022:2003: Informação e documentação - Artigo em publicação 
% periódica científica impressa - Apresentação
% ------------------------------------------------------------------------
% ------------------------------------------------------------------------

\documentclass[
    % -- opções da classe memoir --
    article,            % indica que é um artigo acadêmico
    11pt,               % tamanho da fonte
    oneside,            % para impressão apenas no verso. Oposto a twoside
    a4paper,            % tamanho do papel. 
    % -- opções da classe abntex2 --
    %chapter=TITLE,     % títulos de capítulos convertidos em letras maiúsculas
    %section=TITLE,     % títulos de seções convertidos em letras maiúsculas
    %subsection=TITLE,  % títulos de subseções convertidos em letras maiúsculas
    %subsubsection=TITLE % títulos de subsubseções convertidos em letras maiúsculas
    % -- opções do pacote babel --
    english,            % idioma adicional para hifenização
    brazil,             % o último idioma é o principal do documento
    sumario=tradicional
    ]{abntex2}


% ---
% PACOTES
% ---

% ---
% Pacotes fundamentais 
% ---
\usepackage{lmodern}            % Usa a fonte Latin Modern
\usepackage[T1]{fontenc}        % Selecao de codigos de fonte.
\usepackage[utf8]{inputenc}     % Codificacao do documento (conversão automática dos acentos)
\usepackage{indentfirst}        % Indenta o primeiro parágrafo de cada seção.
\usepackage{nomencl}            % Lista de simbolos
\usepackage{color}              % Controle das cores
\usepackage{graphicx}           % Inclusão de gráficos
\usepackage{microtype}          % para melhorias de justificação
% ---
        
% ---
% Pacotes adicionais, usados apenas no âmbito do Modelo Canônico do abnteX2
% ---
\usepackage{lipsum}             % para geração de dummy text
% ---
        
% ---
% Pacotes de citações
% ---
\usepackage[brazilian,hyperpageref]{backref}     % Paginas com as citações na bibl
\usepackage[alf]{abntex2cite}   % Citações padrão ABNT
% ---

% ---
% Configurações do pacote backref
% Usado sem a opção hyperpageref de backref
\renewcommand{\backrefpagesname}{Citado na(s) página(s):~}
% Texto padrão antes do número das páginas
\renewcommand{\backref}{}
% Define os textos da citação
\renewcommand*{\backrefalt}[4]{
    \ifcase #1 %
        Nenhuma citação no texto.%
    \or
        Citado na página #2.%
    \else
        Citado #1 vezes nas páginas #2.%
    \fi}%
% ---

% ---
% Informações de dados para CAPA e FOLHA DE ROSTO
% ---
\titulo{Construindo um gateway para dispositivos IoT com MEAN stack e MQTT}
\autor{Danilo Guimarães \thanks{guimaraesdjl@gmail.com} \and Ricardo Rodrigues \thanks{ricardo.faria@outlook.com.br} \and Uillan Araújo \thanks{uillan@outlook.com}}
\local{Brasil}
\data{2017, v-1.9.2}
% ---

% ---
% Configurações de aparência do PDF final

% alterando o aspecto da cor azul
\definecolor{blue}{RGB}{41,5,195}

% informações do PDF
\makeatletter
\hypersetup{
        %pagebackref=true,
        pdftitle={\@title}, 
        pdfauthor={\@author},
        pdfsubject={\@title},
        pdfcreator={LaTeX with abnTeX2},
        pdfkeywords={abnt}{latex}{abntex}{abntex2}{atigo científico}, 
        colorlinks=true,            % false: boxed links; true: colored links
        linkcolor=blue,             % color of internal links
        citecolor=blue,             % color of links to bibliography
        filecolor=magenta,              % color of file links
        urlcolor=blue,
        bookmarksdepth=4
}
\makeatother
% --- 

% ---
% compila o indice
% ---
\makeindex
% ---

% ---
% Altera as margens padrões
% ---
\setlrmarginsandblock{3cm}{3cm}{*}
\setulmarginsandblock{3cm}{3cm}{*}
\checkandfixthelayout
% ---

% --- 
% Espaçamentos entre linhas e parágrafos 
% --- 

% O tamanho do parágrafo é dado por:
\setlength{\parindent}{1.3cm}

% Controle do espaçamento entre um parágrafo e outro:
\setlength{\parskip}{0.2cm}  % tente também \onelineskip

% Espaçamento simples
\SingleSpacing

% ----
% Início do documento
% ----
\begin{document}

% Retira espaço extra obsoleto entre as frases.
\frenchspacing 

% ----------------------------------------------------------
% ELEMENTOS PRÉ-TEXTUAIS
% ----------------------------------------------------------

%---
%
% Se desejar escrever o artigo em duas colunas, descomente a linha abaixo
% e a linha com o texto ``FIM DE ARTIGO EM DUAS COLUNAS''.
% \twocolumn[           % INICIO DE ARTIGO EM DUAS COLUNAS
%
%---
% página de titulo
\maketitle

% resumo em português
\begin{resumoumacoluna}
 Conforme a ABNT NBR 6022:2003, o resumo é elemento obrigatório, constituído de
 uma sequência de frases concisas e objetivas e não de uma simples enumeração
 de tópicos, não ultrapassando 250 palavras, seguido, logo abaixo, das palavras
 representativas do conteúdo do trabalho, isto é, palavras-chave e/ou
 descritores, conforme a NBR 6028. (\ldots) As palavras-chave devem figurar logo
 abaixo do resumo, antecedidas da expressão Palavras-chave:, separadas entre si por
 ponto e finalizadas também por ponto.
 
 \vspace{\onelineskip}
 
 \noindent
 \textbf{Palavras-chaves}: latex. abntex. editoração de texto.
\end{resumoumacoluna}

% ]                 % FIM DE ARTIGO EM DUAS COLUNAS
% ---

% ----------------------------------------------------------
% ELEMENTOS TEXTUAIS
% ----------------------------------------------------------
\textual

% ----------------------------------------------------------
%% CAPITULOS
% ----------------------------------------------------------


\chapter{Introdu��o}
\label{cap:intro}

Este documento mostra como usar o \LaTeX\ com a classe \textsf{inf-ufg} para formatar teses, disserta��es, monografias e relat�rios de conclus�o de curso, segundo o padr�o adotado pelo Instituto de Inform�tica da UFG. Este documento e a classe \textsf{inf-ufg} foram, em grande parte, copiados e adaptados da classe \textsf{thesisPUC} e do texto de Thomas Lewiner \cite{Lew2002} que descreve a sua utiliza��o.

 \LaTeX\ � um sistema de editora��o eletr�nica muito usado para produzir documentos cient�ficos de alta qualidade tipogr�fica. O sistema tamb�m � �til para produzir todos os tipos de outros documentos, desde simples cartas at� livros completos.

Se voc� precisar de algum material de apoio referente ao \LaTeX, d� uma olhada em um dos sites do Comprehensive TEX Archive Network (CTAN). O site est� em \href{http://www.ctan.org/}{www.ctan.org}. Todos os pacotes podem ser obtidos via \textsf{FTP} \href{ftp://www.ctan.org/}{ftp://www.ctan.org} e existem v�rios servidores em todo o mundo. Eles podem ser encontrados, por exemplo, em \href{ftp://ctan.tug.org/}{ftp://ctan.tug.org} (EUA), \href{ftp://ftp.dante.de/}{ftp://ftp.dante.de} (Alemanha), \href{ftp://ftp.tex.ac.uk/}{ftp://ftp.tex.ac.uk} (Reino Unido).

Voc� pode encontrar uma grande quantidade de informa��es e dicas na p�gina dos usu�rios brasileiros de \LaTeX\ (\TeX-BR). O endere�o � \href{http://biquinho.furg.br/tex-br/}{http://biquinho.furg.br/tex-br/}.
Tanto no CTAN quanto no \TeX-BR est�o dispon�veis bons documentos em portugu�s sobre o \LaTeX. Em particular no CTAN, est� dispon�vel uma introdu��o bastante completa em portugu�s: \href{http://www.ctan.org/tex-archive/info/lshort/portuguese-BR/lshortBR.pdf}{CTAN:/tex-archive/info/lshort/portuguese-BR/}. No \TeX-BR tamb�m existe um documento com exemplos de uso de \LaTeX\ e de v�rios pacotes: \href{http://biquinho.furg.br/tex-br/doc/LaTeX-demo/}{http://biquinho.furg.br/tex-br/doc/LaTeX-demo/} . O objetivo � ser, atrav�s de exemplos, um guia para o usu�rio de \LaTeX\ iniciante e intermedi�rio, podendo, ainda, servir como um guia de refer�ncia r�pida para usu�rios avan�ados.

Se voc� quer usar o \LaTeX\ em seu computador, verifique em quais sistemas ele est� dispon�vel em \href{http://www.ctan.org/tex-archive/systems/}{CTAN:/tex-archive/systems}. Em particular para \textsf{MS Windows}, o sistema gratuito \href{http://www.miktex.org/}{MikTeX}, dispon�vel no CTAN e no site \href{http://www.miktex.org/}{www.miktex.org} � completo e atualizado de todas as op��es  que voc� poderia precisar para editar o seu texto.

O estilo \textsf{inf-ufg} se integra completamente ao \LaTeXe. Uma tese, disserta��o ou monografia escrita no estilo padr�o do \LaTeX\ para teses (estilo \verb|report|) pode ser formatada em 15 minutos para se adaptar �s normas da UFG.

O estilo \textsf{inf-ufg} foi desenhado para minimizar a quantidade de texto e de comandos necess�rios para escrever a sua disserta��o. S� � preciso inserir algumas macros no in�cio do seu arquivo \LaTeX, precisando os dados bibliogr�ficos da sua disserta��o (por exemplo o seu nome, o titulo da disserta��o\ldots). Em seguida, cada p�gina dos elementos pr�-textuais ser� formatada usando macros ou ambientes espec�ficos. O corpo do texto � editado normalmente. Finalmente, as refer�ncias bibliogr�ficas podem ser entradas manualmente (via o comando \verb|\bibitem| do \LaTeX\ padr�o) ou usando o sistema BiBTeX (muito mais recomend�vel). Neste caso, os arquivos \verb|inf-ufg.bst| e \verb|abnt-alf.bst| permitem a formata��o das refer�ncias bibliogr�ficas segundo as normas da UFG.


\section{Internet of Things}
\label{sec:iot}

A Internet começou como uma forma do governo comunicar após uma guerra nuclear, mas evoluiu para ser muito mais do que uma rede. De muitas maneiras, a Internet tornou-se um mundo digital que tem ligações ao nosso mundo físico \cite{BrasilEscola}.

Conforme as tecnologias avançam, torna-se cada vez mais comum que todos estejam conectados. E com essa evolução os objetos físicos passam a coexistir com a Internet, impactando em diversos aspectos no cotidiano das pessoas seja no profissional ou pessoal.

A Internet das Coisas (do inglês, \textit{Internet of Things}, ou simplesmente IoT) é essa revolução tecnológica que visa conectar dispositivos eletrônicos (como aparelhos eletrodomésticos, máquinas industriais, meios de transporte etc.) à Internet. IoT é um termo criado por Kevin Ashton \cite{Kevin}, um pioneiro da tecnologia britânico que concebeu, em 1999, um sistema de sensores onipresentes conectando o mundo físico à Internet, enquanto trabalhava em identificação por rádio frequência (RFID). O grande valor da IoT está no preenchimento das lacunas entre o mundo físico e digital em sistemas \cite{Amazon}.

Na sua essência, a IoT significa apenas um ambiente que reúne informações de vários dispositivos (computadores, smartphones, semáforos, e quaisquer coisa com um sensor) e de aplicações (qualquer coisa desde uma aplicação de mídia social como o Twitter a uma plataforma de comércio eletrônico, de um sistema de produção a um sistema de controlo de tráfego). Quando se combinam informações de dispositivos e de outros sistemas, enormes recursos de processamento são utilizados para análises expansivas, geralmente associadas com o conceito de \textit{Big Data} \footnote{\textit{Big Data} é um termo usado para referenciar grandes e complexos volumes de informações que necessitam de técnicas e ferramentas específicas para serem capturadas, gerenciadas e/ou processadas} – ou seja, a análise de dados não necessariamente concebidos para serem avaliados em conjunto. Esta noção de múltiplas finalidades é provavelmente a melhor razão para usar o termo “Internet das Coisas”, quando a Internet é mais do que uma rede resistente para ser um canal para qualquer combinação e coleção de atividades digitais \cite{ComputerWorld}.

Em seu processo evolutivo a IoT enfrenta diversos problemas, que variam de aplicativos (sistemas), politicas de segurança e até problemas técnicos. Com todos estes dispositivos conectados à Internet uma enorme quantidade de informação é disponibilizada levantando questões de confiabilidade destas informações. Onde e quem garantirá a autenticidade dessas informações? Quem pode ter acesso à essas informações? Quem irá proteger essas informações? São alguns dos problemas enfrentamos ao se disponibilizar as informações de objetos do mundo físico ao mundo virtual. Uma padronização entre as tecnologias é bastante importante, pois ela levará a uma melhor interoperabilidade, reduzindo barreiras. Muitos fabricantes estão criando suas próprias soluções (Intel, Dell etc.) o que as levam à ter comportamentos diferentes, dificultando a integração destes sistemas ou dispositivos \cite{IEEEORG, CMSWIRE}.
\section{IoT Gateway}
\label{sec:iotGateway}

Quando falamos de IoT, ou Internet das Coisas, já pensamos em que algo estará conectado à Internet. Essa “coisa” não necessariamente se conecta de forma direta, sendo, na grande maioria dos casos, por meio um gateway ou roteador.

O gateway é similar a um roteador, porém, ele pode unir redes de diferentes protocolos, através de um processamento local para a tradução e conversão de protocolos. Gateways são muito utilizados em ambiente industrial e corporativo, porém, com o avanço do IoT, está ficando mais comum encontrar esse tipo de equipamento para uso residencial.

A grande vantagem em fazer o seu próprio gateway, é o nível de customização que ele pode ter, já que temos total acesso ao sistema operacional, sem restrições impostas pelo fabricante – o que ocorre na maioria dos casos. Essa customização vai permitir usar o gateway em modo “fog computing” (computação em nevoeiro), para o processamento de informações o mais perto do dispositivo da borda, ou edge device, fazendo com que, mesmo na falta de Internet, o dispositivo consiga se manter operacional, ainda que com algumas restrições.

Fazer o seu próprio gateway de IoT pode parecer loucura, mas com a baixa nos preços dos SoC’s (System-on-a-Chip), alavancado principalmente pela Raspberry Pi, torna possível a criação de sistemas computacionais de bom desempenho, tamanho reduzido e baixo custo. Já é possível encontrar SoC’s custando menos de US$ 10.



-- Escrever direto sobre Gateway IOT

% ----------------------------------------------------------
% Introdução
% ----------------------------------------------------------


% ---
% Finaliza a parte no bookmark do PDF, para que se inicie o bookmark na raiz
% ---
\bookmarksetup{startatroot}% 
% ---

% ---
% Conclusão
% ---
\section*{Considerações finais}
\addcontentsline{toc}{section}{Considerações finais}

\lipsum[1]

\begin{citacao}
\lipsum[2]
\end{citacao}

\lipsum[3]

% ----------------------------------------------------------
% ELEMENTOS PÓS-TEXTUAIS
% ----------------------------------------------------------
\postextual

% ---
% Título e resumo em língua estrangeira
% ---

% \twocolumn[           % INICIO DE ARTIGO EM DUAS COLUNAS

% titulo em inglês
\titulo{Canonical academic article model with \abnTeX}
\emptythanks
\maketitle

% resumo em português
\renewcommand{\resumoname}{Abstract}
\begin{resumoumacoluna}
 \begin{otherlanguage*}{english}
   According to ABNT NBR 6022:2003, an abstract in foreign language is a back
   matter mandatory element.

   \vspace{\onelineskip}
 
   \noindent
   \textbf{Key-words}: latex. abntex.
 \end{otherlanguage*}  
\end{resumoumacoluna}

% ]                 % FIM DE ARTIGO EM DUAS COLUNAS
% ---

% ----------------------------------------------------------
% Referências bibliográficas
% ----------------------------------------------------------
\bibliography{./bib/artigo}

% ----------------------------------------------------------
% Glossário
% ----------------------------------------------------------
%
% Há diversas soluções prontas para glossário em LaTeX. 
% Consulte o manual do abnTeX2 para obter sugestões.
%
%\glossary

% ----------------------------------------------------------
% Apêndices
% ----------------------------------------------------------

% ---
% Inicia os apêndices
% ---
\begin{apendicesenv}

% ----------------------------------------------------------
\chapter{Nullam elementum urna vel imperdiet sodales elit ipsum pharetra ligula
ac pretium ante justo a nulla curabitur tristique arcu eu metus}
% ----------------------------------------------------------
\lipsum[55-57]

\end{apendicesenv}
% ---

% ----------------------------------------------------------
% Anexos
% ----------------------------------------------------------
\cftinserthook{toc}{AAA}
% ---
% Inicia os anexos
% ---
%\anexos
\begin{anexosenv}

% ---
\chapter{Cras non urna sed feugiat cum sociis natoque penatibus et magnis dis
parturient montes nascetur ridiculus mus}
% ---

\lipsum[31]

\end{anexosenv}

\end{document}