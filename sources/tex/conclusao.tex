\section*{Considerações finais}
\addcontentsline{toc}{section}{Considerações finais}

Durante o desenvolvimento desse trabalho, podemos concluir que a possibilidade de soluções usando IoT é extremamente vasta e extensa. A indústria, como um todo, está aquecida e pretende absorver toda gama de demanda de eventuais serviços que envolvam esse tipo de tecnologia.

Por se tratar de um ramo da Computação que é relativamente novo (o termo foi cunhado em 1999 \cite{Kevin}, somando assim apenas 18 anos de idade no momento que este trabalho foi publicado), algumas definições não possuem uma consistência desejável e costumam divergir muito de autor para autor. O que caracteriza um certo aspecto positivo, tal que ainda há muito espaço para que mais trabalhos possam ser feitos e consolidados na indústria de IoT.

Existem tecnologias Web modernas, principalmente as baseadas em Javascript, que propiciam a construção de softwares voltado para IoT de forma fácil, produtiva, testável e manutenível.

Na elaboração desse projeto, no que tange a perspectiva da escrita do software, o maior desafio foi dos pontos de vista da elaboração arquitetural e de modelagem de dados, de forma que após as definições, a escrita do código foi relativamente simples. Requisitos não funcionais, como os de extensibilidade e de retrocompatibilidade, agregam muito esforço no desenho de um MVP, mas são aspectos que em hipótese alguma podem ser desconsiderados.

