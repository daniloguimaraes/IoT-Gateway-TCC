\section{Considerações finais}
\addcontentsline{toc}{section}{Considerações finais}

Durante o desenvolvimento desse trabalho, podemos concluir que a possibilidade de soluções usando IoT é extremamente vasta e extensa. A indústria, como um todo, está aquecida e pretende absorver toda gama de demanda de eventuais serviços que envolvam esse tipo de tecnologia.

Os Gateways IoT são um ramo da Computação que é relativamente novo, algumas definições não possuem uma consistência desejável e costumam divergir muito de autor para autor. O que caracteriza um certo aspecto positivo, tal que ainda há muito espaço para que mais trabalhos possam ser feitos e consolidados na indústria de IoT.

Existem tecnologias Web modernas, principalmente as baseadas em Javascript, que propiciam a construção de softwares voltado para IoT de forma fácil, produtiva, testável e manutenível.

Na elaboração desse projeto, no que tange a perspectiva da escrita do software, o maior desafio foi dos pontos de vista da elaboração arquitetural e de modelagem de dados, de forma que após as definições, a escrita do código foi relativamente simples.

\subsection{Trabalhos futuros}
\label{trabalhosFuturos}

Como trabalhos futuros, destacamos um suporte a mais protocolos de comunicação com dispositivos e outras funções existentes em outros gateways, como a transmissão posterior das informações armazenadas para a cloud e suporte a novos eventos como, envio de email, ligações e até envio de informações para outros dispositivos, assim tornando o gateway como um dispositivo não só passivo, mas como atuante na rede em que eles está inserido.


Outro aspecto que poderia ser melhorado é o suporte a outros operadores lógicos, tais como, maior que (>), maior ou igual que (\begin{math} \geq \end{math}), menor que (<), menor ou igual que (\begin{math} \leq \end{math}), diferente de (\begin{math}\ne\end{math}) etc.



