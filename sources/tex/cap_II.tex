\section{Internet of Things}
\label{sec:iot}

A Internet das Coisas (do inglês, Internet of Things) é uma revolução tecnológica a fim de conectar dispositivos eletrônicos utilizados no dia-a-dia (como aparelhos eletrodomésticos, eletroportáteis, máquinas industriais, meios de transporte etc.) à Internet, cujo desenvolvimento depende da inovação técnica dinâmica em campos tão importantes como os sensores wireless, a inteligência artificial e a nanotecnologia.

IoT é um termo criado por Kevin Ashton, um pioneiro tecnológico britânico que concebeu um sistema de sensores omnipresentes conectando o mundo físico à Internet, enquanto trabalhava em identificação por rádio frequência (RFID). Embora as Internet, as "coisas" (things) e a conectividade entre elas sejam os três principais componentes da Internet, o valor acrescentado está no preenchimento das lacunas entre os mundos físico e digital em sistemas.

Na sua essência, a IoT significa apenas um ambiente que reúne informações de vários dispositivos (computadores, veículos, smartphones, semáforos, e quase qualquer coisa com um sensor) e de aplicações (qualquer coisa desde uma aplicação de mídia social como o Twitter a uma plataforma de comércio eletrônico, de um sistema de produção a um sistema de controlo de tráfego).

A IoT torna-se interessante quando se combinam informações de dispositivos e de outros sistemas de forma inédita, entrando nos enormes recursos de processamento disponíveis hoje para fazer os tipos de análise expansiva geralmente associada com o conceito de big data – ou seja, a análise de dados não necessariamente concebidos para serem analisados em conjunto.

Esta noção de múltiplas finalidades é provavelmente a melhor razão para usar o termo “Internet das Coisas”, quando a Internet é mais do que uma rede resistente para ser um canal para qualquer combinação e coleção de atividades digitais. A Internet começou como uma forma de o governo comunicar após uma guerra nuclear, mas evoluiu para ser muito mais do que uma rede. De muitas maneiras, a Internet tornou-se um mundo digital que tem ligações ao nosso mundo físico. A IoT eleva esse conceito para o próximo nível, permitindo que vários mundos – alguns ligados a outros, outros não – juntem o físico e o digital de todos os tipos de formas.


-- Escrever direto sobre IOT