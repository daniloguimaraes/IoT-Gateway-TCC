\section{Internet of Things}
\label{sec:iot}

Conforme as tecnologias avançam, torna-se cada vez mais comum que todos estejam conectados. E com essa evolução os objetos físicos passam a coexistir com a Internet, impactando em diversos aspectos no cotidiano das pessoas seja no profissional ou pessoal.

A Internet das Coisas (do inglês, \textit{Internet of Things}, ou simplesmente IoT) é essa revolução tecnológica que visa conectar dispositivos eletrônicos (como aparelhos eletrodomésticos, máquinas industriais, meios de transporte etc.) à Internet. IoT é um termo criado por Kevin Ashton \cite{Kevin}, um pioneiro da tecnologia britânico que concebeu um sistema de sensores onipresentes conectando o mundo físico à Internet, enquanto trabalhava em identificação por rádio frequência (RFID). Embora a IoT e a conectividade entre elas sejam os três principais componentes da Internet, o valor acrescentado está no preenchimento das lacunas entre o mundo físico e digital em sistemas.

Na sua essência, a IoT significa apenas um ambiente que reúne informações de vários dispositivos (computadores, smartphones, semáforos, e quaisquer coisa com um sensor) e de aplicações (qualquer coisa desde uma aplicação de mídia social como o Twitter a uma plataforma de comércio eletrônico, de um sistema de produção a um sistema de controlo de tráfego). Quando se combinam informações de dispositivos e de outros sistemas, enormes recursos de processamento são utilizados para análises expansivas, geralmente associadas com o conceito de \textit{Big Data} \footnote{\textit{Big Data} é um termo usado para referenciar grandes e complexos volumes de informações que necessitam de técnicas e ferramentas específicas para serem capturadas, gerenciadas e/ou processadas} – ou seja, a análise de dados não necessariamente concebidos para serem avaliados em conjunto. Esta noção de múltiplas finalidades é provavelmente a melhor razão para usar o termo “Internet das Coisas”, quando a Internet é mais do que uma rede resistente para ser um canal para qualquer combinação e coleção de atividades digitais. A Internet começou como uma forma do governo comunicar após uma guerra nuclear, mas evoluiu para ser muito mais do que uma rede. De muitas maneiras, a Internet tornou-se um mundo digital que tem ligações ao nosso mundo físico \cite{ComputerWorld}.

Em seu processo evolutivo a IoT enfrenta diversos problemas, que variam de aplicativos (sistemas), politicas de segurança e até problemas técnicos. Com todos estes dispositivos conectados à Internet uma enorme quantidade de informação é disponibilizada levantando questões de confiabilidade destas informações. Onde e quem garantirá a autenticidade dessas informações? Quem pode ter acesso à essas informações? Quem irá proteger essas informações? São alguns dos problemas enfrentamos ao se disponibilizar as informações de objetos do mundo físico ao mundo virtual.

Uma padronização entre as tecnologias é bastante importante, pois ela levará a uma melhor interoperabilidade, reduzindo barreiras. Muitos fabricantes estão criando suas próprias soluções (Intel, Dell etc.) o que as levam à ter comportamentos diferentes, dificultando a integração destes sistemas ou dispositivos.