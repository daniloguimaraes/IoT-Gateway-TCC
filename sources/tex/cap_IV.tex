\section{Solução desenvolvida}
\label{sec:iotGateway}

Além desse artigo, foi desenvolvida também uma solução de software para atingir os objetivos aqui propostos. O projeto está disponibilizado no Github \cite{IoTGatewayGithub} e essa seção se propõe a detalhar as decisões arquiteturais.

As decisões de tecnologia por trás da implementação do projeto foram feitas sempre com foco na portabilidade e execução em dispositivos com baixo recurso de memória e processamento. Portanto, selecionamos o seguinte conjunto de tecnologias:

\begin{itemize}
	\item Node.js \cite{NodeJS}
	\item Typescript \cite{Typescript}
	\item MongoDB \cite{MongoDB}
	\item AngularJS \cite{AngularJS}
\end{itemize}
Onde Node.js, Typescript e MongoDB são aplicados no backend e servidor de eventos, e AngularJS no front end para configuração da aplicação.
O Node.js foi escolhido por conta de seu baixo consumo de memória e processamento, além do seu Non-Blocking IO, garantindo que possamos servir mais clientes com menos recursos. O Typescript veio para trazer tipos e tornar mais fácil o desenvolvimento e manutenção da aplicação Node.js.
O MongoDB despontou entre as outras opções de banco de dados por trazer um serviço leve, de fácil execução em ambientes Linux e armazenamento de grande quantidade de informações, algo que em um futuro onde pretendemos armazenar durante dias os eventos publicados, pode ser de grande valia.
O AngularJS foi escolhido, por ser uma tecnologia que funciona com javascript nativo, sem necessidade de nenhum pós-processador para servir a aplicação, permitindo o seu uso diretamente entre os arquivos estáticos do mesmo servidor Node.js que expõe a aplicação. Além disso, contou como um ponto para a escolha, a experiência prévia da equipe no desenvolvimento com este tipo de tecnologia.

Gateway para IoT
\begin{verbatim}
TypeScript 2.3 com transpile para ES6;
TSLint 4.x com recomentações gerais padrão (*Nota: TSLint >= 5.x ainda não é suportado pelo tslint-microsoft-contrib;
Jest para teste unitário e cobertura;
Definições de tipo para Node.js v6.x (LTS) e Jest;
Angular 1.6 para o front end da aplicação;
\end{verbatim}

Scripts disponíveis
\begin{verbatim}
npm start - Inicia o servidor do express.js;
npm clean - remove todos os caches e arquivos tranpilados;
npm build - Transpile de TypeScript para ES6;
npm run watch - Inicia um watch que efetua o transpile automatico de coisas alteradas;
npm lint - Executa um lint dos arquivos e dos testes;
npm test - Executa os testes;
npm test:watch - Inicia um watch que executa os testes sempre que algo for modificado;
\end{verbatim}

MongoDb, pode ser baixado através de imagem docker e executado da seguinte forma:
\begin{verbatim}
$ docker pull tutum/mongodb
$ docker run -d -p 27017:27017 -p 28017:28017 -e MONGODB_PASS="iot-gateway" tutum/mongodb
\end{verbatim}

Executando a aplicação
Execute o comando npm run watch em um console e deixe aberto
Execute o comando npm start em outro console
Abra o navegador na tela padrão conforme a porta configurada (ela será escrita no log, por padrão é 3000).
Carregando com dados básicos:

Os dados básicos para execução da app, são criados através de uma URL exposta no endpoint de api no endereço: \verb|http://localhost:3000/api/basicos|

Ao executar este endereço, será criado um cadastro wildcard, com ID do MQTT Client '*', de forma que qualquer dispositivo enviando os dados, terá seu cadastro de dispositivo atendido para este.

O evento padrão é o envio de SMS para o número padrão e será enviado sempre que a aplicação receber um valor \verb|true|	.