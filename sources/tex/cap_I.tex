\section*{Introdução}
\addcontentsline{toc}{section}{Introdução}
\label{sec:intro}

Com os recentes avanços das tecnologias, especificamente nas últimas décadas e devido a democratização da Internet, nossa sociedade tem caminhado para um cenário cada vez mais conectado. Se antes apenas super-computadores e máquinas robustas eram conectadas à rede, a tendência nos próximos anos é que dispositivos cada vez menores também tenham seu espaço na Internet.
A essa tendência chamamos \textit{Internet of Things}, ou simplesmente IoT. É uma nova visão que descreve objetos fazendo parte da rede, onde cada um deles é unicamente identificado, acessível através da rede, com posição e estado conhecido, captando informações sensoriais ou agindo sobre o ambiente. Serviços são construídos com base nesses objetos.

Estima-se que até 2050, seja investido cerca de U\$ 19 bi em IoT.

\subsection*{Objetivos}
\label{subsec:objetivos}
O objetivo desse trabalho é construir um IoT Gateway open-source, funcional e utilizável em projetos de pequeno e médio porte.

Não faz parte do objetivos desse trabalho competir, de quaisquer aspectos possíveis, com soluções de IoT Gateway existentes no mercado.

\subsection*{Motivação}
\label{subsec:motivacao}
A principal motivação da realização desse trabalho foi adentrar no assunto de IoT para entender melhor como as tecnologias envolvidas funcionam.

\subsection*{Organização do trabalho}
\label{subsec:organizacao}

Esse trabalho foi organizado da seguinte maneira:

\begin{itemize}
	\item \nameref{sec:intro}
	\item \nameref{sec:iot}
	\item \nameref{sec:iotGateway}
	
\end{itemize}

