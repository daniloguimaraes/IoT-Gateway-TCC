\section*{Introdução}
\addcontentsline{toc}{section}{Introdução}
\label{sec:intro}

Com os recentes avanços das tecnologias, especificamente nas últimas décadas e devido a democratização da Internet, nossa sociedade tem caminhado para um cenário cada vez mais conectado. Se antes apenas super-computadores e máquinas robustas eram conectadas à rede, a tendência nos próximos anos é que dispositivos cada vez menores também tenham seu espaço na Internet.
A essa tendência chamamos \textit{Internet of Things}, ou simplesmente IoT. É uma nova visão que descreve objetos fazendo parte da rede, onde cada um deles é unicamente identificado, acessível através da rede, com posição e estado conhecido, captando informações sensoriais ou agindo sobre o ambiente. Serviços são construídos com base nesses objetos.

Estima-se que até 2050, seja investido cerca de U\$ 19 bi em IoT.

A principal motivação da realização desse trabalho foi adentrar no assunto de IoT para entender melhor como as tecnologias envolvidas funcionam.
-- Escrever mais a motivação

O objetivo desse trabalho é construir um IoT Gateway open-source, funcional e utilizável em projetos de pequeno e médio porte.

As decisões de tecnologia por trás da implementação do projeto, foram feitas sempre com foco na portabilidade e execução em dispositivos com baixo recurso de memória e processamento, portanto selecionamos o seguinte set de tecnologias:
\begin{itemize}
\item Node.js
\item Typescript
\item MongoDB
\item AngularJS
\end{itemize}
Onde Node.js, Typescript e MongoDB são aplicados no backend e servidor de eventos, e AngularJS no front end para configuração da aplicação.
O Node.js foi escolhido por conta de seu baixo consumo de memória e processamento, além do seu Non-Blocking IO, garantindo que possamos servir mais clientes com menos recursos. O Typescript veio para trazer tipos e tornar mais fácil o desenvolvimento e manutenção da aplicação Node.js.
O MongoDB despontou entre as outras opções de banco de dados por trazer um serviço leve, de fácil execução em ambientes Linux e armazenamento de grande quantidade de informações, algo que em um futuro onde pretendemos armazenar durante dias os eventos publicados, pode ser de grande valia.
O AngularJS foi escolhido, por ser uma tecnologia que funciona com javascript nativo, sem necessidade de nenhum pós-processador para servir a aplicação, permitindo o seu uso diretamente entre os arquivos estáticos do mesmo servidor Node.js que expõe a aplicação. Além disso, contou como um ponto para a escolha, a experiência prévia da equipe no desenvolvimento com este tipo de tecnologia.

Não faz parte do objetivos desse trabalho competir, de quaisquer aspectos possíveis, com soluções de IoT Gateway existentes no mercado.
-- Escrever mais sobre os objetivos

