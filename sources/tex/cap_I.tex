\chapter{Introdu��o}
\label{cap:intro}

Com os recentes avan�os das tecnologias, especificamente nas �ltimas d�cadas e devido a democratiza��o da Internet, nossa sociedade tem caminhado para um cen�rio cada vez mais conectado. Se antes apenas super-computadores e m�quinas robustas eram conectadas � rede, a tend�ncia nos pr�ximos anos � que dispositivos cada vez menores tamb�m tenham seu espa�o na Internet.
A essa tend�ncia chamamos \textit{Internet of Things}, ou simplesmente IoT. � uma nova vis�o que descreve objetos fazendo parte da rede, onde cada um deles � unicamente identificado, acess�vel atrav�s da rede, com posi��o e estado conhecido, captando informa��es sensoriais ou agindo sobre o ambiente. Servi�os s�o constru�dos com base nesses objetos.

Estima-se que at� 2050, seja investido cerca de U\$ 19 bi em IoT.

\section{Objetivos}
\label{sec:objetivos}
O objetivo desse trabalho � construir um IoT Gateway open-source, funcional e utiliz�vel em projetos de pequeno e m�dio porte.

N�o faz parte do objetivos desse trabalho competir, de quaisquer aspectos poss�veis, com solu��es de IoT Gateway existentes no mercado.

\section{Motiva��o}
\label{sec:motivacao}
A principal motiva��o da realiza��o desse trabalho foi adentrar no assunto de IoT para entender melhor como as tecnologias envolvidas funcionam.

\section{Organiza��o do trabalho}
\label{sec:organizacao}

Esse trabalho foi organizado da seguinte maneira:

\begin{itemize}
	\item Cap�tulo \ref{cap:intro}: introdu��o, objetivos e motiva��o.
	\item Cap�tulo \ref{cap:descr}: 
	
\end{itemize}

